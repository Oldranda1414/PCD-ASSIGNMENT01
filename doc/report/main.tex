\documentclass[12pt, a4paper]{report}
\usepackage[pdftex]{graphicx} %for embedding images
\usepackage[english,italian]{babel}
\usepackage{url} %for proper url entries
% \usepackage[bookmarks, colorlinks=false, pdfborder={0 0 0}, pdftitle={<pdf title here>}, pdfauthor={<author's name here>}, pdfsubject={<subject here>}, pdfkeywords={<keywords here>}]{hyperref} %for creating links in the pdf version and other additional pdf attributes, no effect on the printed document
%\usepackage[final]{pdfpages} %for embedding another pdf, remove if not required

\begin{document}
\renewcommand\bibname{References} %Renames "Bibliography" to "References" on ref page


\begin{titlepage}

\begin{center}

\Large \textbf {Programmazione Concorrente e Distribuita - Assigment 01}\\%\\[0.5in]
\vspace{1em}%
\vfill
Leonardo Randacio


Filippo Gurioli


Andrea Biagini
\vspace{1em}
\vfill
{\bf Università di Bologna \\ Scienze e Ingegneria Informatiche}\\[0.5in]

       
\vfill
\today

\end{center}

\end{titlepage}


\tableofcontents
\listoffigures
\listoftables

\newpage
\pagenumbering{arabic} %reset numbering to normal for the main content

\chapter{Analysis}
The goal is to create a concurrent agent-based simulation.

An agent-based simulation or model is a computational modeling
 technique used to simulate complex systems by representing individual
 entities, known as agents, and their interactions within an environment.
 In the goal model all agents have identical behavior.
 Agents beheviour can be described in 3 fases:
 \begin{itemize}
    \item sense fase: the agent acquires data from the environment
    \item decide fase: the agent determines the next action
    \item act fase: the action determined is executed on the environment
 \end{itemize}



\chapter{Design}

% \input{./prob-definition.tex} %objective changed to problem definition
\bibliographystyle{plain}
\bibliography{References}

\end{document}